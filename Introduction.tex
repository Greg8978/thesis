\ifdefined\included
\else
\documentclass[a4paper,11pt,twoside]{StyleThese}
\usepackage{amsmath,amssymb}             % AMS Math
\usepackage[french]{babel}
\usepackage[utf8]{inputenc}
\usepackage[T1]{fontenc}
\usepackage{tabularx}
%\usepackage{tabular}
\usepackage{multirow}


\usepackage[tight,footnotesize]{subfigure}
\usepackage{algorithm} %To allow algorithm environment
\usepackage{algpseudocode} %Provides algorithmic environment

\usepackage{hhline}
\usepackage[left=1.5in,right=1.3in,top=1.1in,bottom=1.1in,includefoot,includehead,headheight=13.6pt]{geometry}
\renewcommand{\baselinestretch}{1.05}

% Table of contents for each chapter

\usepackage[nottoc, notlof, notlot]{tocbibind}
\usepackage[french]{minitoc}
\setcounter{minitocdepth}{2}
\mtcindent=15pt
% Use \minitoc where to put a table of contents

\usepackage{aecompl}

% Glossary / list of abbreviations

\usepackage[intoc]{nomencl}
\renewcommand{\nomname}{Liste des Abréviations}

\makenomenclature

% My pdf code

\usepackage{ifpdf}

\ifpdf
  \usepackage[pdftex]{graphicx}
  \DeclareGraphicsExtensions{.jpg}
  \usepackage[a4paper,pagebackref,hyperindex=true]{hyperref}
  \usepackage{tikz}
  \usetikzlibrary{arrows,shapes,calc}
\else
  \usepackage{graphicx}
  \DeclareGraphicsExtensions{.ps,.eps}
  \usepackage[a4paper,dvipdfm,pagebackref,hyperindex=true]{hyperref}
\fi

\graphicspath{{.}{images/}}

%nicer backref links
\renewcommand*{\backref}[1]{}
\renewcommand*{\backrefalt}[4]{%
\ifcase #1 %
(Non cité.)%
\or
(Cité en page~#2.)%
\else
(Cité en pages~#2.)%
\fi}
\renewcommand*{\backrefsep}{, }
\renewcommand*{\backreftwosep}{ et~}
\renewcommand*{\backreflastsep}{ et~}

% Links in pdf
\usepackage{color}
\definecolor{linkcol}{rgb}{0,0,0.4} 
\definecolor{citecol}{rgb}{0.5,0,0} 
\definecolor{linkcol}{rgb}{0,0,0} 
\definecolor{citecol}{rgb}{0,0,0}
% Change this to change the informations included in the pdf file

\hypersetup
{
bookmarksopen=true,
pdftitle="Évaluation de la sécurité des équipements grand public connectés à Internet",
pdfauthor="Yann BACHY", %auteur du document
pdfsubject="Thèse", %sujet du document
%pdftoolbar=false, %barre d'outils non visible
pdfmenubar=true, %barre de menu visible
pdfhighlight=/O, %effet d'un clic sur un lien hypertexte
colorlinks=true, %couleurs sur les liens hypertextes
pdfpagemode=None, %aucun mode de page
pdfpagelayout=SinglePage, %ouverture en simple page
pdffitwindow=true, %pages ouvertes entierement dans toute la fenetre
linkcolor=linkcol, %couleur des liens hypertextes internes
citecolor=citecol, %couleur des liens pour les citations
urlcolor=linkcol %couleur des liens pour les url
}

% definitions.
% -------------------

\setcounter{secnumdepth}{3}
\setcounter{tocdepth}{2}

% Some useful commands and shortcut for maths:  partial derivative and stuff

\newcommand{\pd}[2]{\frac{\partial #1}{\partial #2}}
\def\abs{\operatorname{abs}}
\def\argmax{\operatornamewithlimits{arg\,max}}
\def\argmin{\operatornamewithlimits{arg\,min}}
\def\diag{\operatorname{Diag}}
\newcommand{\eqRef}[1]{(\ref{#1})}

\usepackage{rotating}                    % Sideways of figures & tables
%\usepackage{bibunits}
%\usepackage[sectionbib]{chapterbib}          % Cross-reference package (Natural BiB)
%\usepackage{natbib}                  % Put References at the end of each chapter
                                         % Do not put 'sectionbib' option here.
                                         % Sectionbib option in 'natbib' will do.
\usepackage{fancyhdr}                    % Fancy Header and Footer

% \usepackage{txfonts}                     % Public Times New Roman text & math font
  
%%% Fancy Header %%%%%%%%%%%%%%%%%%%%%%%%%%%%%%%%%%%%%%%%%%%%%%%%%%%%%%%%%%%%%%%%%%
% Fancy Header Style Options

\pagestyle{fancy}                       % Sets fancy header and footer
\fancyfoot{}                            % Delete current footer settings

%\renewcommand{\chaptermark}[1]{         % Lower Case Chapter marker style
%  \markboth{\chaptername\ \thechapter.\ #1}}{}} %

%\renewcommand{\sectionmark}[1]{         % Lower case Section marker style
%  \markright{\thesection.\ #1}}         %

\fancyhead[LE,RO]{\bfseries\thepage}    % Page number (boldface) in left on even
% pages and right on odd pages
\fancyhead[RE]{\bfseries\nouppercase{\leftmark}}      % Chapter in the right on even pages
\fancyhead[LO]{\bfseries\nouppercase{\rightmark}}     % Section in the left on odd pages

\let\headruleORIG\headrule
\renewcommand{\headrule}{\color{black} \headruleORIG}
\renewcommand{\headrulewidth}{1.0pt}
\usepackage{colortbl}
\arrayrulecolor{black}

\fancypagestyle{plain}{
  \fancyhead{}
  \fancyfoot{}
  \renewcommand{\headrulewidth}{0pt}
}

%\usepackage{MyAlgorithm}
%\usepackage[noend]{MyAlgorithmic}
\usepackage[ED=MITT - STICIA, Ets=INP]{tlsflyleaf}
%%% Clear Header %%%%%%%%%%%%%%%%%%%%%%%%%%%%%%%%%%%%%%%%%%%%%%%%%%%%%%%%%%%%%%%%%%
% Clear Header Style on the Last Empty Odd pages
\makeatletter

\def\cleardoublepage{\clearpage\if@twoside \ifodd\c@page\else%
  \hbox{}%
  \thispagestyle{empty}%              % Empty header styles
  \newpage%
  \if@twocolumn\hbox{}\newpage\fi\fi\fi}

\makeatother
 
%%%%%%%%%%%%%%%%%%%%%%%%%%%%%%%%%%%%%%%%%%%%%%%%%%%%%%%%%%%%%%%%%%%%%%%%%%%%%%% 
% Prints your review date and 'Draft Version' (From Josullvn, CS, CMU)
\newcommand{\reviewtimetoday}[2]{\special{!userdict begin
    /bop-hook{gsave 20 710 translate 45 rotate 0.8 setgray
      /Times-Roman findfont 12 scalefont setfont 0 0   moveto (#1) show
      0 -12 moveto (#2) show grestore}def end}}
% You can turn on or off this option.
% \reviewtimetoday{\today}{Draft Version}
%%%%%%%%%%%%%%%%%%%%%%%%%%%%%%%%%%%%%%%%%%%%%%%%%%%%%%%%%%%%%%%%%%%%%%%%%%%%%%% 

\newenvironment{maxime}[1]
{
\vspace*{0cm}
\hfill
\begin{minipage}{0.5\textwidth}%
%\rule[0.5ex]{\textwidth}{0.1mm}\\%
\hrulefill $\:$ {\bf #1}\\
%\vspace*{-0.25cm}
\it 
}%
{%

\hrulefill
\vspace*{0.5cm}%
\end{minipage}
}

\let\minitocORIG\minitoc
\renewcommand{\minitoc}{\minitocORIG \vspace{1.5em}}

\usepackage{multirow}
%\usepackage{slashbox}

\newenvironment{bulletList}%
{ \begin{list}%
	{$\bullet$}%
	{\setlength{\labelwidth}{25pt}%
	 \setlength{\leftmargin}{30pt}%
	 \setlength{\itemsep}{\parsep}}}%
{ \end{list} }

\newtheorem{definition}{Définition}
\renewcommand{\epsilon}{\varepsilon}

% centered page environment

\newenvironment{vcenterpage}
{\newpage\vspace*{\fill}\thispagestyle{empty}\renewcommand{\headrulewidth}{0pt}}
{\vspace*{\fill}}

\usepackage{tablefootnote}
\sloppy
\begin{document}
\fi


\chapter*{Introduction}
\addstarredchapter{Introduction}
\minitoc

\section{Contexte général}
%global intro
De nombreux fantasmes ont toujours entouré la représentation que l'on se fait des robots. Le concept de créatures intelligentes confectionnées par la main de l'homme est déjà présent dans les mythologies tel que le mythe du Golem dans la mythologie juive ou l'histoire de Pygmalion et Galatée dans la mythologie grecque, ou encore les servantes androïdes en or du dieu Héphaïstos que l'on retrouve dans \textit{l'Illiade} d'Homère.

On retrouve dans la littérature du XIXe siècle le thème de l'automate prenant vie, à travers des oeuvres tel que \textit{L'homme au sable} d'Ernst Theodor Amadeus Hoffmann ou le compte de fées \textit{pinocchio} de Carlo Collodi. C'est également à cette époque qu'est paru le célèbre roman de Mary Shelley: \textit{Frankenstein ou le Prométhée moderne}. Cette oeuvre met en scène la perte de contrôle du savant Frankeinstein sur sa créature, cette dernière se révolte contre son créateur et les êtres humains en général dont il est rejeté. 

Cette thématique de révolte contre l'homme sera très largement reprise dans les oeuvres de science-fiction occidentales du XXème siècle. Un auteur cependant se démarque de cette ligne en présentant un reccueil de nouvel nommé \textit{Les Robots}. Il s'agit de l'auteur américain Isaac Asimov, qui à travers ces nouvelles présente les limites possibles de système robotique et comment s'assurer que les robots s'évertuent à contribuer au bien-être des humains en énonçant notamment les trois lois de la robotique. Ces lois, implantées au coeur même du système robotique, doivent forcer le robot à agir pour garantir l'intégrité physique de tout être humain, pour obéir aux ordres des hommes et enfin pour garantir sa propre intégrité physique.

De nos jours, la thématique de la machine échappant au contrôle de l'homme est toujours présente notamment dans le cinéma Hollywoodien à travers les oeuvres tel que Terminator, The Machine ou encore Ex-Machina. Il est cependant à constater que ces machines ont des comportements de plus en plus proche de l'être humain.  de par leur intelligence mais aussi leur capacités d'établir des interactions sociales  avec l'homme.
Certaines oeuvres cependant, dans la lignée d'Asimov, décrivent les robots comme des compagnons particulièrement utiles et dociles, tel que le pinocchio moderne David du film \textit{AI} ou le robot médical Baymax du film d'animation \textit{les nouveaux héros}.

La représentation que l'opinion publique se fait des robots a son importance car elle influence la perception des robots \cite{Sundar2016} et donc la direction donnée à la recherche. Ainsi, la culture japonaise est décrite comme robophile \cite{gilson98}. Dans ce même pays, on constate un investissement massif dans la recherche en robotique, notamment dans la robotique de service depuis plusieurs décennies.

Malgré les scénarios apocalyptiques de certaines oeuvres et les angoisses sociétales liées au robot, tel la suppression d’emplois peu qualifiés, les robots sont entrés dans nos usines et commencent à arriver dans nos maisons. Les premiers modèles ont pris la forme d'automates programmables dans les industries et de robots aspirateurs ou de jouets dans les foyers. Ces premières formes ont une intelligence très limitée et liée à une tâche, le plus souvent assez basique, à accomplir. Ces premiers modèles, de par leur adaptabilité très limitée et leur intelligence qui tient plus de la réaction que du raisonnement, sont plus proches de l'automate que  de réelles systèmes intelligents. Cependant, on constate depuis quelques années une complexification des robots industriels et domestiques. Cette complexification permets aux robots les plus récents de prendre en compte l'homme. Ainsi, l'homme et la machine vont être amenés de plus en plus à se cotoyer, que se soit sur le lieu de travail avec des robot coéquipiers ou dans les foyers avec les robots assistants. C'est dans ce contexte que les robots sociaux arrivent sur le marché. Ainsi, plusieurs campagnes de crowd-founding ont rencontré un franc succès pour des projets de plateformes robotiques sociales pour le domicile. On peut notamment citer les robots Jibo\footnote{https://www.jibo.com/} ou Buddy\footnote{http://www.bluefrogrobotics.com/fr/buddy-fr/}. Ces deux projets de plateforme robotique interagissant avec l'homme et destinée au grand public, ont été un franc succés et sont révélateurs de l'engouement de la population pour ce genre de plateforme. En effet, le secteur de la robotique de service et domestique (robots assistants/équipiers) est considéré comme un des enjeux
économiques de ce siècle. On estime que d’ici à 2020, le marché de la robotique de services (tous secteurs confondus) pourrait représenter un volume supérieur à 15.69 milliards de dollars par an selon une étude de Grand View Research\footnote{http://www.grandviewresearch.com/industry-analysis/service-robotics-industry}. 

%Vieillissement de la population
L'un des intérêts premier de la robotique d'assistance est lié au vieillissement de la population. En effet, les robots autonomes d'assistance pourraient permettre de redonner une certaine autonomie aux personnes âgées ainsi qu'un accès aux technologies modernes à travers l'utilisation intuitive des robots comme interface.
Pour ce qui est des robots équipiers, ils pourraient permettre d'exécuter des tâches dangereuses voir irréalisables par l'homme tout en travaillant en collaboration avec un opérateur humain. 

Cependant, pour le robot coéquipier comme pour le robot assistant, le contact avec l'homme rend important, si ce n'est nécessaire, d'incorporer une représentation de l'homme et des comportements sociaux pour interagir avec celui-ci. Le but n'étant pas de copier l'homme ou de le remplacer, mais uniquement de mettre en place des mécanismes permettant une interaction efficace, agréable et intuitive pour l'homme. Pour cela le robot doit être capable d'estimer la situation et l'état de l'homme et exhiber des comportements pouvant être compris par l'homme. Dans cette thèse nous présenterons comment le robot peut créer et maintenir une représentation de l'environment dans lequel il évolue ainsi que des individus avec lesquels il intéragit, afin de pouvoir montrer des comportements sociallement acceptables par l'homme.



\section{Motivations}
%Challenges
%Maybe start with a scenario?
%Big focus on HRI


\subsection{Scénario}

La robotique d'assistance et plus généralement la recherche concernant la conception et la confection de systèmes interagissant avec l'homme a encore de nombreux défits à relever. Pour exposer certains de ces défis, nous proposons un scénario d'illustration.

Imaginons un couple, Bob et Alice, vivant dans un appartement avec un robot d'assistance. Prenons la situation où Alice rentre du travaille et décide de cuisiner une nouvelle recette avec l'aide de son robot. Avant de commencer la préparation, Alice demande à son robot de l'aider à réunir les ingrédients et ustensiles pour faire la recette.
Pour pouvoir aider efficacement Alice, le robot a donc besoin de capacités élémentaires tel que la navigation, la perception et la préhension d'objets.

Par exemple, Alice peut demander au robot d'apporter un objet qui est sur la table du salon. Pour comprendre Alice, à supposer que le robot soit capable de reconnaissance vocale, il lui faut comprendre également les concepts énoncés par Alice et les relier à la réalité du contexte situé. Ceci implique pour le robot de raisonner sur l'environnement afin de générer ce genre de relations spatiales (ici un objet \textit{"sur"} un autre). Ceci afin de comprendre et d'être compris de l'humain avec lequel il interagit.

Imaginons à présent qu'Alice ait besoin de son batteur à oeufs qui est dans la commode du salon. Imaginons également que pendant la journée, Bob ait déplacé le batteur de la commode à l'armoire du salon, et que le robot ait perçu ce changement.
Un comportement utile et proactif pour le robot serait, en voyant Alice se déplacer vers la commode du salon, de la prévenir du changement de position du batteur.
Pour parvenir à cela, le robot doit être capable de se représenter les états de connaissances des humains qui l'entourent ainsi que d'interpréter leurs intentions en fonction du contexte (ici chercher les ustensiles pour réaliser une recette), de leur état mental (ici Alice pense que le batteur est dans la commode) et de leurs actions (Alice se dirige vers la commode du salon).

Une fois que les ustensiles et les ingrédients sont réunis, il reste à confectionner le plat. Imaginons qu'Alice demande au robot de la guider pour effectuer une tarte aux pommes. Le robot doit alors être capable de générer un plan collaboratif prenant en compte les compétences d'Alice mais aussi pouvoir négocier ce plan avec Alice en fonction de ses préférences et des capacités d'execution de chacun.
Pour ne pas être perçu comme ennuyeux, le robot doit également pouvoir adapter son niveau de détail explicatif au niveau de compétence d'Alice sur les tâches du plan qui lui incombent.
Cela nécessite pour le robot d'avoir une représentation et une capacité de suivi des connaissances des humains concernant les tâches à accomplir.


% Mettre que le scénario dans la partie précédente et lister les capacités nécéssaires dans la partie suivante?
%\subsection{Compétences requises}


\subsection{Défis liés à la thèse et contributions associées}
%TODO
Parmi les défis soulevés par le scénario présenté dans la partie précédente, certains seront traités dans cette thèse.

%énoncer les défis dans l'ordre du plan
Le premier défi est de permettre au robot d'obtenir une représentation de l'environnement qui l'entoure. La contribution n'est pas au niveau de la perception proprement dite mais au niveau des raisonnements mis en place pour permettre au robot, à partir de l'agrégation des données de perception, de maintenir une représentation tridimensionnelle du monde qui l'entoure et des différentes entités présentes (objets, humains et robots). À partir de cette représentation tridimensionnelle de l'environnement maintenue par le robot, celui-ci est capable  de mettre en oeuvre divers raisonnements spatio-temporels permettant d'estimer la situation de l'environnement. Cette couche de représentation symbolique de l'état du monde permet de réduire l'écart entre les données de perception (sub-symboliques) avec la couche décisionnelle (la supervision).

La seconde contribution reliée également à cette estimation de la situation est la mise en place d'un modèle d'état mental pour chaque agent de l'interaction. Ainsi, en plus de connaître la situation de l'environnement, le robot est capable d'estimer la situation pour les autres. Cette capacité, appelée prise de perspective dans la litterature psychologique, est une capacité essentielle pour de nombreux aspects de l'interaction entre agents sociaux. Il est donc crucial pour le robot d'être lui aussi doté de ce type de capacité cognitive.

En plus de ces deux contributions scientifiques liées à l'estimation de la situation, nous avons développé une infrastructure logicielle nommée TOASTER (Tracking Of Agent and Spatio-TEmporal Reasonning)\footnote{http://www.gregoire.milliez.fr/toaster/index.html}. TOASTER est l'implémentation des deux contributions scientifiques décrites ci-dessus. Il est disponible en Open-Source et a été conçu de façon générique afin de profiter au plus grand nombre. En effet les calculs et raisonnements réalisés sont indépendants des données d'entrées (des capteurs) et de la plateforme robotique.

L'établissement d'une couche symbolique représentant l'état du monde permet au robot de mettre en place un dialogue situé de qualité. Cette contribution est complétée par une étude montrant comment nous avons amélioré l'éfficacité et le taux de succés du dialogue situé en utilisant l'état mental de l'homme pour mieux comprendre ses propos.


Pour aller plus loin dans l'estimation de la situation de l'homme, et basé sur son état mental modélisé et maintenu par la capacité de prise de perspective, nous avons mis en place un mécanisme permettant d'estimer l'intention de l'homme. Cette contribution est complétée par l'utilisation qui est faite de cette estimation de l'intention pour donner au robot un comportement proactif afin d'aider au mieux l'humain. 

Le dernier défi relevé par cette thèse est de mettre en place une modélisation de l'état mental de l'humain, cette fois-ci au niveau des connaissances qu'il peut avoir concernant certaines tâches d'un plan partagé. Cette modélisation permet d'adapter la génération du plan collaboratif et l'execution de ce plan au niveau d'expertise de l'humain.

%\section{Contributions associées}
%Maintenant que nous avons présenter les défis de cette thèse, nous présentons ici les contributions apportées durant la thèse, tant au niveau des contributions scientifiques à travers le développement de nouveau concepts et algorithmes qu'au niveau téchniques à travers le développement d'outils informatiques.


%\subsection{Contributions Scientifiques}


%\subsection{Contributions Techniques}

\section{Publications}

Les publications suivantes sont liées aux contributions scientifiques de cette thèse:

\begin{itemize}
\item \textbf{Grégoire Milliez}, Raphaël Lallement, Michelangelo Fiore et Rachid Alami, \textit{"Using human knowledge awareness to adapt collaborative plan generation, explanation and monitoring"}, The Eleventh ACM/IEEE International Conference on Human Robot Interaction, (\textbf{HRI 2016})
\item Michelangelo Fiore, Harmish Khambhaita, \textbf{Grégoire Milliez} et Rachid Alami, \textit{"An Adaptive and Proactive Human-Aware Robot Guide"}, Social Robotics, (\textbf{ICSR 2015})
\item Emmanuel Ferreira, \textbf{Grégoire Milliez}, Fabrice Lefèvre et Rachid Alami, \textit{"Users’ Belief Awareness in Reinforcement Learning-Based Situated Human–Robot Dialogue Management"}, International Workshop on Spoken Dialog Systems, (\textbf{IWSDS 2015})
\item \textbf{Grégoire Milliez}, Emmanuel Ferreira, Michelangelo Fiore, Rachid Alami et Fabrice Lefèvre, \textit{"Simulating human-robot interactions for dialogue strategy learning"}, Simulation, Modeling, and Programming for Autonomous Robots, (\textbf{SIMPAR 2014})
\item Séverin Lemaignan, Marc Hanheide, Michael Karg, Harmish Khambhaita, Lars Kunze, Florian Lier, Ingo Lütkebohle et \textbf{Grégoire Milliez}, \textit{"Simulation and HRI recent perspectives with the MORSE simulator"}, Simulation, Modeling, and Programming for Autonomous Robots, (\textbf{SIMPAR 2014})
\item \textbf{Grégoire Milliez}, Matthieu Warnier, Aurélie Clodic et Rachid Alami, \textit{"A framework for endowing an interactive robot with reasoning capabilities about perspective-taking and belief management"}, Robot and Human Interactive Communication, 2014 RO-MAN: The 23rd IEEE (\textbf{ROMAN 2014})
\end{itemize}

	
%TODO? add workshops?	
	
	
%	Some essential skills and their combination in an architecture for a cognitive and interactive robot
%S Devin, G Milliez, M Fiore, A Clodic, R Alami
%arXiv preprint arXiv:1603.00583	 	2016
%	Planning Human-Robot Interaction Tasks Using Graph Models
%LJ Manso, P Bustos, R Alami, G Milliez, P Núnez
 	


\section{Organisation du manuscrit}
Dans ce manuscrit, nous présentons tout d'abord dans le chapitre \ref{chapter1}, comment notre système est capable  de construire et maintenir une représentation de l'état du monde. Nous montrerons également l'évaluation de la situation de l'environnement et des agents faite à partir de calculs spatio-temporels sur l'état du monde qu'il maintient.
Dans un second temps, pour aller plus loin dans la compréhension de la situation des humains, nous expliquerons dans le chapitre \ref{chapter2} comment nous avons doté notre robot de prise de perspective.
Par la suite nous expliquerons dans le chapitre \ref{chapter3} comment l'évaluation de la situation permet d'établir un dialogue situé avec l'homme, et en quoi la capacité de gérer explicitement des croyances divergentes permet d'améliorer la qualité de l'interaction et la compréhension de l'homme par le robot.
Dans le chapitre \ref{chapter4} nous montrerons comment la connaissance de la situation et la possibilité de raisonner en se mettant à la place de l'homme permet une reconnaissance d'intentions appropriées de celui-ci et comment nous avons pu grâce à cela doter notre robot de comportements proactifs pour venir en aide à l'homme.
Pour finir, nous présenterons en chapitre \ref{chapter5} une étude présentant un système de maintien d'un modèle des connaissances de l'homme sur diverses tâches et qui permet une gestion adaptée de l'interaction lors de l'élaboration interactive et l'accomplissement d'un plan partagé.

\ifdefined\included
\else
\bibliographystyle{acm}
\bibliography{These}
\end{document}
\fi
