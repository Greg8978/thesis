\ifdefined\included
\else
\documentclass[a4paper,11pt,twoside]{StyleThese}
\include{formatAndDefs}
\sloppy
\begin{document}
\fi


\chapter*{Abstract}
\addstarredchapter{Abstract} %Sinon cela n'apparait pas dans la table des matières

The first robots appeared in factories, 
in the form of programmable controllers.
These first robotic forms usually had a 
very limited number of sensors and simply 
repeated a small set of sequences of motions and actions.

Nowadays, more and more robots have to interact
 or cooperate with humans, whether at the workplace 
 with teammate robots or at home with assistance robots.

Introducing a robot in a human environment raises many challenges.
Indeed, to evolve in the same environment as humans, and to understand 
this environment, the robot must be equipped with appropriate cognitive abilities.

Beyond understanding the physical environment, the robot must 
be able to reason about human partners in order to work with 
them or serve them best. When the robot interacts with humans, 
the fulfillment of the task is not a sufficient criterion to 
quantify the quality of the interaction. Indeed, as the human is a 
social being, it is important that the robot can have reasoning 
mechanisms allowing it to assess the mental state of the human
to improve his understanding and efficiency, but also to exhibit 
social behaviors in order to be accepted and to ensure the comfort of the human.

In this manuscript, we first present a generic framework (independent of the robotic platform and 
sensors used) to build and maintain a representation 
of the state of the world by using the aggregation of data entry 
and hypotheses on the environment. This infrastructure is also 
in charge of assessing the situation. Using the state of the world it maintains, 
the system is able to utilize 
various spatio-temporal reasoning to assess the situation of the environment 
and the situation of the present agents (humans and robots). This allows 
the creation and maintenance of a symbolic representation of the state 
of the world and to keep awareness of each agent status.

Second, to go further in understanding the situation of the humans, 
we will explain how we designed our robot with the capacity known in 
developmental and cognitive psychology as "theory of mind",
embodied here by mechanisms allowing the system to reason by putting itself in the 
human situation, that is to be equipped with "perspective-taking" ability.
Later we will explain how the assessment of the situation enables 
a situated dialogue with the human, and how the ability to explicitly 
manage conflicting beliefs can improve the quality of interaction 
and understanding of the human by the robot.
We will also show how knowledge of the situation and the perspective taking ability allows proper recognition 
of human intentions and how we enhanced the robot 
with proactive behaviors to help the human.
Finally, we present a study where a system maintains a human model 
of knowledge on various tasks to improve the management of the 
interaction during the interactive development and fulfillment of a shared plan.
 

\ifdefined\included
\else
\bibliographystyle{acm}
\bibliography{These}
\end{document}
\fi