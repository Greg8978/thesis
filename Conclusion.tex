\ifdefined\included
\else
\documentclass[a4paper,11pt,twoside]{StyleThese}
\usepackage{amsmath,amssymb}             % AMS Math
\usepackage[french]{babel}
\usepackage[utf8]{inputenc}
\usepackage[T1]{fontenc}
\usepackage{tabularx}
%\usepackage{tabular}
\usepackage{multirow}


\usepackage[tight,footnotesize]{subfigure}
\usepackage{algorithm} %To allow algorithm environment
\usepackage{algpseudocode} %Provides algorithmic environment

\usepackage{hhline}
\usepackage[left=1.5in,right=1.3in,top=1.1in,bottom=1.1in,includefoot,includehead,headheight=13.6pt]{geometry}
\renewcommand{\baselinestretch}{1.05}

% Table of contents for each chapter

\usepackage[nottoc, notlof, notlot]{tocbibind}
\usepackage[french]{minitoc}
\setcounter{minitocdepth}{2}
\mtcindent=15pt
% Use \minitoc where to put a table of contents

\usepackage{aecompl}

% Glossary / list of abbreviations

\usepackage[intoc]{nomencl}
\renewcommand{\nomname}{Liste des Abréviations}

\makenomenclature

% My pdf code

\usepackage{ifpdf}

\ifpdf
  \usepackage[pdftex]{graphicx}
  \DeclareGraphicsExtensions{.jpg}
  \usepackage[a4paper,pagebackref,hyperindex=true]{hyperref}
  \usepackage{tikz}
  \usetikzlibrary{arrows,shapes,calc}
\else
  \usepackage{graphicx}
  \DeclareGraphicsExtensions{.ps,.eps}
  \usepackage[a4paper,dvipdfm,pagebackref,hyperindex=true]{hyperref}
\fi

\graphicspath{{.}{images/}}

%nicer backref links
\renewcommand*{\backref}[1]{}
\renewcommand*{\backrefalt}[4]{%
\ifcase #1 %
(Non cité.)%
\or
(Cité en page~#2.)%
\else
(Cité en pages~#2.)%
\fi}
\renewcommand*{\backrefsep}{, }
\renewcommand*{\backreftwosep}{ et~}
\renewcommand*{\backreflastsep}{ et~}

% Links in pdf
\usepackage{color}
\definecolor{linkcol}{rgb}{0,0,0.4} 
\definecolor{citecol}{rgb}{0.5,0,0} 
\definecolor{linkcol}{rgb}{0,0,0} 
\definecolor{citecol}{rgb}{0,0,0}
% Change this to change the informations included in the pdf file

\hypersetup
{
bookmarksopen=true,
pdftitle="Évaluation de la sécurité des équipements grand public connectés à Internet",
pdfauthor="Yann BACHY", %auteur du document
pdfsubject="Thèse", %sujet du document
%pdftoolbar=false, %barre d'outils non visible
pdfmenubar=true, %barre de menu visible
pdfhighlight=/O, %effet d'un clic sur un lien hypertexte
colorlinks=true, %couleurs sur les liens hypertextes
pdfpagemode=None, %aucun mode de page
pdfpagelayout=SinglePage, %ouverture en simple page
pdffitwindow=true, %pages ouvertes entierement dans toute la fenetre
linkcolor=linkcol, %couleur des liens hypertextes internes
citecolor=citecol, %couleur des liens pour les citations
urlcolor=linkcol %couleur des liens pour les url
}

% definitions.
% -------------------

\setcounter{secnumdepth}{3}
\setcounter{tocdepth}{2}

% Some useful commands and shortcut for maths:  partial derivative and stuff

\newcommand{\pd}[2]{\frac{\partial #1}{\partial #2}}
\def\abs{\operatorname{abs}}
\def\argmax{\operatornamewithlimits{arg\,max}}
\def\argmin{\operatornamewithlimits{arg\,min}}
\def\diag{\operatorname{Diag}}
\newcommand{\eqRef}[1]{(\ref{#1})}

\usepackage{rotating}                    % Sideways of figures & tables
%\usepackage{bibunits}
%\usepackage[sectionbib]{chapterbib}          % Cross-reference package (Natural BiB)
%\usepackage{natbib}                  % Put References at the end of each chapter
                                         % Do not put 'sectionbib' option here.
                                         % Sectionbib option in 'natbib' will do.
\usepackage{fancyhdr}                    % Fancy Header and Footer

% \usepackage{txfonts}                     % Public Times New Roman text & math font
  
%%% Fancy Header %%%%%%%%%%%%%%%%%%%%%%%%%%%%%%%%%%%%%%%%%%%%%%%%%%%%%%%%%%%%%%%%%%
% Fancy Header Style Options

\pagestyle{fancy}                       % Sets fancy header and footer
\fancyfoot{}                            % Delete current footer settings

%\renewcommand{\chaptermark}[1]{         % Lower Case Chapter marker style
%  \markboth{\chaptername\ \thechapter.\ #1}}{}} %

%\renewcommand{\sectionmark}[1]{         % Lower case Section marker style
%  \markright{\thesection.\ #1}}         %

\fancyhead[LE,RO]{\bfseries\thepage}    % Page number (boldface) in left on even
% pages and right on odd pages
\fancyhead[RE]{\bfseries\nouppercase{\leftmark}}      % Chapter in the right on even pages
\fancyhead[LO]{\bfseries\nouppercase{\rightmark}}     % Section in the left on odd pages

\let\headruleORIG\headrule
\renewcommand{\headrule}{\color{black} \headruleORIG}
\renewcommand{\headrulewidth}{1.0pt}
\usepackage{colortbl}
\arrayrulecolor{black}

\fancypagestyle{plain}{
  \fancyhead{}
  \fancyfoot{}
  \renewcommand{\headrulewidth}{0pt}
}

%\usepackage{MyAlgorithm}
%\usepackage[noend]{MyAlgorithmic}
\usepackage[ED=MITT - STICIA, Ets=INP]{tlsflyleaf}
%%% Clear Header %%%%%%%%%%%%%%%%%%%%%%%%%%%%%%%%%%%%%%%%%%%%%%%%%%%%%%%%%%%%%%%%%%
% Clear Header Style on the Last Empty Odd pages
\makeatletter

\def\cleardoublepage{\clearpage\if@twoside \ifodd\c@page\else%
  \hbox{}%
  \thispagestyle{empty}%              % Empty header styles
  \newpage%
  \if@twocolumn\hbox{}\newpage\fi\fi\fi}

\makeatother
 
%%%%%%%%%%%%%%%%%%%%%%%%%%%%%%%%%%%%%%%%%%%%%%%%%%%%%%%%%%%%%%%%%%%%%%%%%%%%%%% 
% Prints your review date and 'Draft Version' (From Josullvn, CS, CMU)
\newcommand{\reviewtimetoday}[2]{\special{!userdict begin
    /bop-hook{gsave 20 710 translate 45 rotate 0.8 setgray
      /Times-Roman findfont 12 scalefont setfont 0 0   moveto (#1) show
      0 -12 moveto (#2) show grestore}def end}}
% You can turn on or off this option.
% \reviewtimetoday{\today}{Draft Version}
%%%%%%%%%%%%%%%%%%%%%%%%%%%%%%%%%%%%%%%%%%%%%%%%%%%%%%%%%%%%%%%%%%%%%%%%%%%%%%% 

\newenvironment{maxime}[1]
{
\vspace*{0cm}
\hfill
\begin{minipage}{0.5\textwidth}%
%\rule[0.5ex]{\textwidth}{0.1mm}\\%
\hrulefill $\:$ {\bf #1}\\
%\vspace*{-0.25cm}
\it 
}%
{%

\hrulefill
\vspace*{0.5cm}%
\end{minipage}
}

\let\minitocORIG\minitoc
\renewcommand{\minitoc}{\minitocORIG \vspace{1.5em}}

\usepackage{multirow}
%\usepackage{slashbox}

\newenvironment{bulletList}%
{ \begin{list}%
	{$\bullet$}%
	{\setlength{\labelwidth}{25pt}%
	 \setlength{\leftmargin}{30pt}%
	 \setlength{\itemsep}{\parsep}}}%
{ \end{list} }

\newtheorem{definition}{Définition}
\renewcommand{\epsilon}{\varepsilon}

% centered page environment

\newenvironment{vcenterpage}
{\newpage\vspace*{\fill}\thispagestyle{empty}\renewcommand{\headrulewidth}{0pt}}
{\vspace*{\fill}}

\usepackage{tablefootnote}
\sloppy
\begin{document}
\fi


\chapter*{Conclusion et Perspectives}
\addstarredchapter{Conclusion et Perspectives} %Sinon cela n'apparait pas dans la table des matières

\section{Conclusion}
Ce manuscrit de thèse rapporte à travers cinqs chapitres comment il est possible d'aquérir et d'utiliser des données contextuelles de différents niveaux d'abstraction pour permettre d'améliorer l'interaction homme-robot dans différents domaines tel que la dialogue situé, l'assistance proactive, la génération de plan, l'execution de tâches collaboratives...

Dans le chapitre \ref{chapter1} nous avons montrer comment il est possible de mettre en place une architecture modulaire basée l'agrégation de données capteurs et sur certains raisonnements pour aquérir et maintenir un état du monde tridimensionel correspondant à la représentation du monde réel tel qu'il est perçu et inféré (grâce à une gestion d'hypothèses) par le robot. Basé sur cette représentation tridimensionnelle, nous montrons comment les différents modules de l'architecture permettent de générer des \textit{faits} constituant une représentation symbolique du monde. Ces faits sont centralisés par une base de donée temporelle qui possède une table permettant de garder en mémoire les faits passés et les transitions survenues. Nous présentons également dans ce premier chapitre TOASTER, une infrastructure logicielle modulaire Open-Source implémentant les principes et modèles énoncés.
Deux exemples d'expérimentations utilisant l'infrastructure logicielle TOASTER comme module d'estimation de la situation ont été présentés comme exemple d'utilisation possible.

Dans le chapitre \ref{chapter2}, nous montrons expliquons brièvement les concepts de prise de perspective perceptuelle et conceptuelle. Nous montrons également que ces capacités sont importantes dans les interactions sociales humaines. Nous présentons en suite comment, à partir de calculs géométriques il nous est possible de doter le système robotique de prise de perspective perceptuelle afin que le robot puisse savoir ce qui est perceptible ou atteignable par l'homme.
Basé sur cette prise de perspective perceptuelle et sur un raisonnement permettant de gérer la mise à jour des croyances des agents, nous avons montrer qu'il est possible de maintenir un état de croyance distinct pour chaque agent présents dans la scène. Cette gestion des croyances permet de doter le robot de la capacité de prise de perspective conceptuelle.
Pour montrer la capacité de notre robot à se mettre à la place d'un autre agent et de raisonner sur ses croyances, nous avons fait passer deux tests à notre robot. Le premier étant le test connu dans la littérature de la philosophie du développement sous le nom de test de Sally et Anne. Le second est un scénario d'interaction où deux hommes manipulent des objets en présence du robot.

Le chapitre \ref{chapter3} décrit comment les données contextuelles et la capacité de prise de perspective permet de mettre en place un dialogue situé de qualité. Nous avons montré comment les données contextuelles de différents niveaux d'abstraction (position, distance, représentation symbolique, prise de perspective) sont utilisés tout au long du processus de dialogue (compréhension de la parole, interprétation des gestes de l'homme, identification du référent, choix de la politique par la couche décisionnelle, choix de la modalité de sortie et modulation de la voix, exploration et execution de la tâche).
Nous présentons notamment une étude menée sur simulateur qui a permis de montrer l'utilité de la prise de perspective conceptuelle pour rendre le dialogue plus efficace et plus précis. Nous présentons également une expérimentation menée sur plateforme robotique qui illustre divers stratégies prenant en compte le contexte et l'état mental de l'homme pour adapter l'execution de la tâche.


\section{Améliorations et travaux à venir}
\ifdefined\included
\else
\bibliographystyle{acm}
\bibliography{These}
\end{document}
\fi